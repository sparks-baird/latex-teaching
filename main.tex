\documentclass{article}

\usepackage{booktabs}
\usepackage{hyperref}
\usepackage{amsmath}
\usepackage[nameinlink,capitalise]{cleveref}
\usepackage[shortcuts,abbreviations,automake]{glossaries-extra}
\newabbreviation{ml}{ML}{machine learning}


\makeglossaries

\title{Cartesian closed categories and the price of eggs}
\author{Jane Doe}
\date{September 1994}
\begin{document}
   \maketitle
   Hello world!

We present breakdown time data for various research activities in \cref{tab:breakdown}.
% Please add the following required packages to your document preamble:
% \usepackage{booktabs}
\begin{table}[h!]
\centering
\caption{Example of breakdown of time spent.}
\label{tab:breakdown}
\begin{tabular}{@{}lll@{}}
\toprule
Task                             & Time Spent (minutes) & Time We Wish We Spent (minutes) \\ \midrule
Fiddling with equations          & 120                  & 5                               \\
Correcting reference order       & 60                   & 0                               \\
Looking for "that one" reference & 90                   & 1                               \\
Copy-pasting figures             & 20                   & 0                               \\
Rerunning all data               & 180                  & 20                              \\
Fixing reference meta-data       & 60                   & 0                               \\
Actually doing research          & 900                  & 1500                            \\ \bottomrule
\end{tabular}
\end{table}

An example of an equation is Fick's second law (\cref{eq:fick-second}).
\begin{equation} \label{eq:fick-second}
    \frac{\partial \varphi}{\partial t}=D \frac{\partial^{2} \varphi}{\partial x^{2}}
\end{equation}





Once we've defined abbreviations in \texttt{abbrev.tex}, we can call those abbreviations using the \texttt{gls} command. In the case of ``machine learning'', the first usage of \texttt{gls} will give the full form as in \gls{ml}. The next usage of \texttt{gls} will give the abbreviated version as in \gls{ml}.

\printglossaries

\end{document}

